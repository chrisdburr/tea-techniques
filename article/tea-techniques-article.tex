% Options for packages loaded elsewhere
\PassOptionsToPackage{unicode}{hyperref}
\PassOptionsToPackage{hyphens}{url}
\documentclass[
  british,
  10pt]{article}
\usepackage{xcolor}
\usepackage[margin=2.5cm]{geometry}
\usepackage{amsmath,amssymb}
\setcounter{secnumdepth}{5}
\usepackage{iftex}
\ifPDFTeX
  \usepackage[T1]{fontenc}
  \usepackage[utf8]{inputenc}
  \usepackage{textcomp} % provide euro and other symbols
\else % if luatex or xetex
  \usepackage{unicode-math} % this also loads fontspec
  \defaultfontfeatures{Scale=MatchLowercase}
  \defaultfontfeatures[\rmfamily]{Ligatures=TeX,Scale=1}
\fi
\usepackage{lmodern}
\ifPDFTeX\else
  % xetex/luatex font selection
\fi
% Use upquote if available, for straight quotes in verbatim environments
\IfFileExists{upquote.sty}{\usepackage{upquote}}{}
\IfFileExists{microtype.sty}{% use microtype if available
  \usepackage[]{microtype}
  \UseMicrotypeSet[protrusion]{basicmath} % disable protrusion for tt fonts
}{}
\makeatletter
\@ifundefined{KOMAClassName}{% if non-KOMA class
  \IfFileExists{parskip.sty}{%
    \usepackage{parskip}
  }{% else
    \setlength{\parindent}{0pt}
    \setlength{\parskip}{6pt plus 2pt minus 1pt}}
}{% if KOMA class
  \KOMAoptions{parskip=half}}
\makeatother
\usepackage{color}
\usepackage{fancyvrb}
\newcommand{\VerbBar}{|}
\newcommand{\VERB}{\Verb[commandchars=\\\{\}]}
\DefineVerbatimEnvironment{Highlighting}{Verbatim}{commandchars=\\\{\}}
% Add ',fontsize=\small' for more characters per line
\newenvironment{Shaded}{}{}
\newcommand{\AlertTok}[1]{\textcolor[rgb]{1.00,0.00,0.00}{\textbf{#1}}}
\newcommand{\AnnotationTok}[1]{\textcolor[rgb]{0.38,0.63,0.69}{\textbf{\textit{#1}}}}
\newcommand{\AttributeTok}[1]{\textcolor[rgb]{0.49,0.56,0.16}{#1}}
\newcommand{\BaseNTok}[1]{\textcolor[rgb]{0.25,0.63,0.44}{#1}}
\newcommand{\BuiltInTok}[1]{\textcolor[rgb]{0.00,0.50,0.00}{#1}}
\newcommand{\CharTok}[1]{\textcolor[rgb]{0.25,0.44,0.63}{#1}}
\newcommand{\CommentTok}[1]{\textcolor[rgb]{0.38,0.63,0.69}{\textit{#1}}}
\newcommand{\CommentVarTok}[1]{\textcolor[rgb]{0.38,0.63,0.69}{\textbf{\textit{#1}}}}
\newcommand{\ConstantTok}[1]{\textcolor[rgb]{0.53,0.00,0.00}{#1}}
\newcommand{\ControlFlowTok}[1]{\textcolor[rgb]{0.00,0.44,0.13}{\textbf{#1}}}
\newcommand{\DataTypeTok}[1]{\textcolor[rgb]{0.56,0.13,0.00}{#1}}
\newcommand{\DecValTok}[1]{\textcolor[rgb]{0.25,0.63,0.44}{#1}}
\newcommand{\DocumentationTok}[1]{\textcolor[rgb]{0.73,0.13,0.13}{\textit{#1}}}
\newcommand{\ErrorTok}[1]{\textcolor[rgb]{1.00,0.00,0.00}{\textbf{#1}}}
\newcommand{\ExtensionTok}[1]{#1}
\newcommand{\FloatTok}[1]{\textcolor[rgb]{0.25,0.63,0.44}{#1}}
\newcommand{\FunctionTok}[1]{\textcolor[rgb]{0.02,0.16,0.49}{#1}}
\newcommand{\ImportTok}[1]{\textcolor[rgb]{0.00,0.50,0.00}{\textbf{#1}}}
\newcommand{\InformationTok}[1]{\textcolor[rgb]{0.38,0.63,0.69}{\textbf{\textit{#1}}}}
\newcommand{\KeywordTok}[1]{\textcolor[rgb]{0.00,0.44,0.13}{\textbf{#1}}}
\newcommand{\NormalTok}[1]{#1}
\newcommand{\OperatorTok}[1]{\textcolor[rgb]{0.40,0.40,0.40}{#1}}
\newcommand{\OtherTok}[1]{\textcolor[rgb]{0.00,0.44,0.13}{#1}}
\newcommand{\PreprocessorTok}[1]{\textcolor[rgb]{0.74,0.48,0.00}{#1}}
\newcommand{\RegionMarkerTok}[1]{#1}
\newcommand{\SpecialCharTok}[1]{\textcolor[rgb]{0.25,0.44,0.63}{#1}}
\newcommand{\SpecialStringTok}[1]{\textcolor[rgb]{0.73,0.40,0.53}{#1}}
\newcommand{\StringTok}[1]{\textcolor[rgb]{0.25,0.44,0.63}{#1}}
\newcommand{\VariableTok}[1]{\textcolor[rgb]{0.10,0.09,0.49}{#1}}
\newcommand{\VerbatimStringTok}[1]{\textcolor[rgb]{0.25,0.44,0.63}{#1}}
\newcommand{\WarningTok}[1]{\textcolor[rgb]{0.38,0.63,0.69}{\textbf{\textit{#1}}}}
\usepackage{longtable,booktabs,array}
\usepackage{calc} % for calculating minipage widths
% Correct order of tables after \paragraph or \subparagraph
\usepackage{etoolbox}
\makeatletter
\patchcmd\longtable{\par}{\if@noskipsec\mbox{}\fi\par}{}{}
\makeatother
% Allow footnotes in longtable head/foot
\IfFileExists{footnotehyper.sty}{\usepackage{footnotehyper}}{\usepackage{footnote}}
\makesavenoteenv{longtable}
\usepackage{graphicx}
\makeatletter
\newsavebox\pandoc@box
\newcommand*\pandocbounded[1]{% scales image to fit in text height/width
  \sbox\pandoc@box{#1}%
  \Gscale@div\@tempa{\textheight}{\dimexpr\ht\pandoc@box+\dp\pandoc@box\relax}%
  \Gscale@div\@tempb{\linewidth}{\wd\pandoc@box}%
  \ifdim\@tempb\p@<\@tempa\p@\let\@tempa\@tempb\fi% select the smaller of both
  \ifdim\@tempa\p@<\p@\scalebox{\@tempa}{\usebox\pandoc@box}%
  \else\usebox{\pandoc@box}%
  \fi%
}
% Set default figure placement to htbp
\def\fps@figure{htbp}
\makeatother
% definitions for citeproc citations
\NewDocumentCommand\citeproctext{}{}
\NewDocumentCommand\citeproc{mm}{%
  \begingroup\def\citeproctext{#2}\cite{#1}\endgroup}
\makeatletter
 % allow citations to break across lines
 \let\@cite@ofmt\@firstofone
 % avoid brackets around text for \cite:
 \def\@biblabel#1{}
 \def\@cite#1#2{{#1\if@tempswa , #2\fi}}
\makeatother
\newlength{\cslhangindent}
\setlength{\cslhangindent}{1.5em}
\newlength{\csllabelwidth}
\setlength{\csllabelwidth}{3em}
\newenvironment{CSLReferences}[2] % #1 hanging-indent, #2 entry-spacing
 {\begin{list}{}{%
  \setlength{\itemindent}{0pt}
  \setlength{\leftmargin}{0pt}
  \setlength{\parsep}{0pt}
  % turn on hanging indent if param 1 is 1
  \ifodd #1
   \setlength{\leftmargin}{\cslhangindent}
   \setlength{\itemindent}{-1\cslhangindent}
  \fi
  % set entry spacing
  \setlength{\itemsep}{#2\baselineskip}}}
 {\end{list}}
\usepackage{calc}
\newcommand{\CSLBlock}[1]{\hfill\break\parbox[t]{\linewidth}{\strut\ignorespaces#1\strut}}
\newcommand{\CSLLeftMargin}[1]{\parbox[t]{\csllabelwidth}{\strut#1\strut}}
\newcommand{\CSLRightInline}[1]{\parbox[t]{\linewidth - \csllabelwidth}{\strut#1\strut}}
\newcommand{\CSLIndent}[1]{\hspace{\cslhangindent}#1}
\ifLuaTeX
\usepackage[bidi=basic,shorthands=off,]{babel}
\else
\usepackage[bidi=default,shorthands=off,]{babel}
\fi
\ifLuaTeX
  \usepackage{selnolig} % disable illegal ligatures
\fi
\setlength{\emergencystretch}{3em} % prevent overfull lines
\providecommand{\tightlist}{%
  \setlength{\itemsep}{0pt}\setlength{\parskip}{0pt}}
\usepackage{xcolor}
\usepackage{bookmark}
\IfFileExists{xurl.sty}{\usepackage{xurl}}{} % add URL line breaks if available
\urlstyle{same}
\hypersetup{
  pdftitle={TEA Techniques: An Interactive Database for Trustworthy and Ethical AI Assurance},
  pdfauthor={Christopher Burr; Levan Bokeria},
  pdflang={en-GB},
  pdfkeywords={AI assurance, trustworthy AI, responsible AI, ethical
AI, explainable AI, AI fairness},
  hidelinks,
  pdfcreator={LaTeX via pandoc}}

\title{TEA Techniques: An Interactive Database for Trustworthy and
Ethical AI Assurance}
\author{Christopher Burr \and Levan Bokeria}
\date{30th July 2025}

\begin{document}
\maketitle
\begin{abstract}
In this paper we present Trustworthy and EThical Assurance (TEA)
Techniques, an interactive and community-centred dataset containing
nearly 100 curated techniques for assuring artificial intelligence (AI)
systems. The dataset organises techniques around the assurance goals
they support, including explainability, fairness, safety, security, and
more. It also provides structured metadata through an extensible tag
system (e.g.~`applicable models', `expertise needed'). To ensure each
technique provides actionable guidance to AI practitioners and
researchers, high-quality resources are also made available
(e.g.~official software packages, journal articles, tutorials). We
explain how these resources were discovered through a systematic
resource discovery pipeline, and also set out our plans for ensuring the
interactive application will remain up-to-date as new advances and
methods emerge. The platform is made freely available and open source as
a static web application to provide AI professionals, practitioners, and
researchers with immediate access to actionable techniques for building
robust AI assurance cases. The TEA Techniques database represents a
significant step towards building a more inclusive and flourishing AI
assurance ecosystem, fostering community-driven development of
responsible AI practices.
\end{abstract}

\section{Introduction}\label{sec:introduction}

The deployment of artificial intelligence (AI) systems across key
sectors of critical national infrastructure (CNI) demands an urgent and
systematic approach to evidencing their trustworthiness and ethical
alignment. While considerable progress has been made in developing
principles and frameworks for responsible AI (Brundage et al. 2020; Raji
et al. 2020), practitioners continue to struggle with identifying and
implementing concrete techniques that can generate meaningful evidence
for AI assurance claims. For instance, many now know that AI can create
and exacerbate biases and discriminatory impacts, but may struggle to
know which technique they should use to mitigate these biases and how to
appropriately communicate the processes to affected stakeholder or
impacted users.

This gap between high-level normative principles and actionable methods
is well recognised, and represents a significant barrier to the
widespread adoption of responsible AI practices. Fortunately, a
considerable amount of research effort has gone into developing
techniques for different AI assurance techniques. But now the challenge
is how to navigate the slightly fragmented landscape.

Techniques for explainable AI, bias assessment and mitigation, privacy
preservation, and other assurance goals are scattered across academic
literature, software repositories, technical blogs, and proprietary
documentation. Practitioners seeking to build comprehensive assurance
cases must navigate this dispersed knowledge base, often lacking the
time or expertise to evaluate the relevance and quality of available
resources or to appropriately determine whether the technique they have
identified is the best one for their specific use case. Moreover, the
rapid evolution of AI technologies means that new techniques emerge
constantly, and existing methods may become outdated or superseded.

To address these challenges, we present \emph{TEA Techniques}: an
interactive database designed to improve access to practical methods for
evidencing claims about responsible AI design, development, and
deployment. Currently, the database contains approximately 100
techniques\footnote{At the time of submisison, the exact number is 92.
  However, we report an approximate number as this figure is likely to
  change over time.} organised around seven core assurance goals:
explainability, fairness, privacy, reliability, safety, security, and
transparency.

Each technique is presented with a description, example use cases, set
of limitations, list of related techniques, external sources
(e.g.~software packages and tutorials) and further enriched with
structured metadata (i.e. tags).

In this paper we motivate and introduce the TEA techniques dataset,
explaining its contribution to the AI assurance ecosystem, and how we
hope it will be leveraged and enhanced by the AI community.
Specifically, the TEA techniques dataset has been designed with the
following intended contributions in mind:

\begin{enumerate}
\def\labelenumi{\arabic{enumi}.}
\tightlist
\item
  \textbf{A curated and structured dataset of AI assurance techniques}
  approximately 100 AI assurance techniques with rich metadata,
  addressing the fragmentation of knowledge in this domain.
\item
  \textbf{An open, accessible platform} that provides both interactive
  web access and programmatic interfaces, facilitating integration into
  diverse practitioner workflows.
\item
  \textbf{A foundation for community-driven development} of AI assurance
  methods, with planned features for user contributions, extensibility,
  and collaborative refinement.
\end{enumerate}

The remainder of this paper is organised as follows. Section
\ref{sec:background} situates TEA Techniques within existing AI
assurance frameworks and related repositories. Section \ref{sec:dataset}
describes the dataset structure, core assurance goals, tagging system,
and interactive web application architecture. Section
\ref{sec:discovery} details the systematic methodology and automated
pipeline used to identify and curate high-quality resources. Section
\ref{sec:technical} covers the static site generation approach and
API-like data access. Section \ref{sec:usecases} presents practitioner
workflows and application scenarios. Section \ref{sec:future} discusses
plans for community contributions, platform extensibility, and
addressing current dataset imbalances. Finally, Section
\ref{sec:conclusion} reflects on the platform's contributions and
potential impact on the AI assurance ecosystem.

\section{Background and Related Work}\label{sec:background}

The field of AI assurance has emerged from the convergence of multiple
disciplines, each contributing unique perspectives on how to ensure AI
systems operate in a way that promotes and builds trust. This section
situates TEA Techniques within the broader landscape of responsible AI
initiatives, examining existing frameworks, resources, and the specific
gaps our work addresses.

\subsection{AI Assurance and Trustworthiness
Frameworks}\label{sec:frameworks}

The concept of AI assurance draws heavily from established practices in
safety-critical systems engineering and safety case development. It is
from this context that argument-based assurance cases have long been
used to demonstrate system properties (Burr and Leslie 2022).

\colorbox{yellow}{TODO: Add more citations to GSN and York's work}

These structured arguments link evidential artefacts (e.g.~reports, test
results) to specific claims about some property of a system or process,
providing clear reasoning about how an overarching goal has been
realised. However, applying argument-based assurance to AI system
design, development, and deployment presents unique challenges due to
several interlocking factors, such as their probabilistic nature,
opacity, and potential for emergent behaviours.

Several high-profile initiatives have attempted to establish guidance
for trustworthy AI, specifically using the language of \emph{AI
assurance}.

The UK's Department for Science, Innovation and Technology (DSIT)
published comprehensive guidance on AI assurance in February 2024,
providing an accessible introduction to assurance mechanisms and global
technical standards (\textbf{dsit2024introduction?}). This guidance
supports the UK's pro-innovation approach to AI regulation, establishing
key concepts and terms within the broader AI governance landscape.

Australia has developed a National Framework for AI Assurance in
Government, established through collaboration between federal and state
governments (\textbf{dta2024framework?}). This framework establishes
nationally consistent, principles-based approaches to AI assurance that
prioritise human oversight and the rights and wellbeing of communities,
with specific assessment processes for high-impact AI use cases.

\colorbox{yellow}{TODO: Mention other key initiatives (US NIST, EU, Singapore) - highlight "risk management" over "assurance"}

\subsection{Existing Technique Collections and
Databases}\label{existing-technique-collections-and-databases}

In addition to high-level frameworks, various efforts have been made to
catalogue AI techniques relevant to assurance goals.

\subsubsection{General Purpose Examples}\label{general-purpose-examples}

\begin{enumerate}
\def\labelenumi{\arabic{enumi}.}
\tightlist
\item
  The AI Incident Database documents failures and harms from deployed AI
  systems, providing valuable lessons but not proactive techniques for
  prevention. \colorbox{yellow}{TODO: add citation}
\item
  The Partnership on AI's publication repository contains numerous
  reports and best practices, though these are primarily narrative
  documents rather than structured, actionable techniques.
  \colorbox{yellow}{TODO: add citation}
\item
  The UK Government's Portfolio of AI Assurance Techniques, developed by
  the Responsible Technology Adoption Unit (RTA), showcases real-world
  examples of AI assurance techniques across multiple sectors
  (\textbf{ukgov2024portfolio?}). This portfolio maps techniques to the
  principles outlined in the UK government's AI regulation white paper,
  illustrating how different approaches can support wider AI governance.
\item
  The OECD/GPAI Catalogue of Tools and Metrics for Trustworthy AI
  provides a comprehensive platform where AI practitioners worldwide can
  discover and share tools, metrics, and use cases for developing
  trustworthy AI systems (\textbf{oecd2024catalogue?}). The catalogue
  encompasses technical tools, metrics for measuring AI trustworthiness,
  and real-world applications across fairness, transparency,
  explainability, robustness, safety, and security.
\end{enumerate}

\subsubsection{Goal-Specific Examples}\label{goal-specific-examples}

\begin{enumerate}
\def\labelenumi{\arabic{enumi}.}
\tightlist
\item
  IBM's AI Fairness 360 (AIF360) offers an extensible open-source
  library containing over 70 fairness metrics and bias mitigation
  algorithms across the AI lifecycle, available in both Python and R
  (\textbf{bellamy2019ai?}). The toolkit includes preprocessing
  approaches like reweighing and disparate impact removal, in-processing
  methods such as adversarial debiasing, and post-processing techniques
  like equalized odds adjustment.
\item
  The AI alignment research community maintains distributed resources
  through organisations like Anthropic (developing introspection
  techniques for model state interpretation), Stanford AI Safety
  (creating responsible AI assessment frameworks), and FAR.AI
  (facilitating technical breakthroughs in frontier alignment research).
  \colorbox{yellow}{TODO: add citations} These efforts focus on
  mechanistic interpretability tools, chain-of-thought faithfulness
  detection, and adversarial evaluation techniques for ensuring model
  alignment with human values.
\item
  Academic surveys have attempted to systematise knowledge in specific
  domains---for instance, comprehensive reviews of explainable AI
  methods (Ribeiro, Singh, and Guestrin 2016; Lundberg and Lee 2017;
  Linardatos, Papastefanopoulos, and Kotsiantis 2020) or fairness
  metrics (Mehrabi et al. 2019; Barocas, Hardt, and Narayanan 2019).
  However, due to the nature of academic publications, these resources
  are static snapshots that will not keep pace with rapid technical
  developments.
\item
  The Centre for Assuring Autonomy at the University of York has
  developed the''BIG Argument'' (Balanced, Integrated, and Grounded) for
  AI safety case (\textbf{habli2025big?}), representing a world-first
  comprehensive safety argument framework. The BIG Argument adopts a
  whole-system approach addressing safety primarily, but alongside
  ethical considerations (e.g.~justice) and integrating social, ethical,
  and technical considerations.
\end{enumerate}

\subsection{The Challenge of Resource Quality and
Curation}\label{the-challenge-of-resource-quality-and-curation}

A persistent challenge in the AI assurance ecosystem is the variable
quality and accessibility of resources. For instance, academic papers
provide rigorous foundations but may lack implementation
details.\footnote{Notable exceptions include papers that provide
  comprehensive implementation guidance, such as Lundberg \& Lee's SHAP
  paper (Lundberg and Lee 2017) which includes detailed code examples,
  and Ribeiro et al.'s LIME work (Ribeiro, Singh, and Guestrin 2016)
  which released accompanying software packages alongside publication.
  \colorbox{yellow}{TODO: check papers with code for other
  examples}} Whereas, code repositories offer practical tools but may
lack theoretical grounding or proper documentation. The former may be
useful for researchers who are looking to explore and learn about
narrowly focused techniques, but are unlikely to be of much use to an AI
practitioner within a commercial organisation who is hoping to meet
specific regulatory requirements for market approval or licensing.
Similarly, blog posts and tutorials can provide accessible explanations
for practitioners and professionals who are not expert in any one area,
but are also likely to oversimplify or misrepresent complex techniques
(e.g.~not presenting limitations). This heterogeneity makes it difficult
for AI practitioners to identify trustworthy, relevant resources for
their specific needs.

Previous attempts at systematic curation have typically relied on manual
expert review, which whilst ensuring quality, limits scalability and
struggles with the volume of new publications and tools. The exponential
growth in AI research---with thousands of new papers published
monthly---makes purely manual curation increasingly untenable. This
motivates our hybrid approach combining automated discovery with
structured quality assessment.

\subsection{Gaps Addressed by TEA
Techniques}\label{gaps-addressed-by-tea-techniques}

Despite the above frameworks, repositories, surveys, and resource
collections, there are still several critical gaps:

\begin{enumerate}
\def\labelenumi{\arabic{enumi}.}
\tightlist
\item
  \textbf{Fragmentation across domains}: existing resources typically
  focus on single assurance goals (e.g., explainability or reliability)
  rather than providing integrated and systematic access to techniques
  spanning multiple objectives.
\item
  \textbf{Lack of structured metadata}: most technique descriptions lack
  consistent but flexible categorisation schemas (e.g.~which models the
  technique is applicable to, which data types are valid, the lifecycle
  stages in which the technique should be used), and other dimensions
  crucial for practical selection.
\item
  \textbf{Missing quality indicators}: AI practitioners have limited
  information for assessing the maturity, reliability, and
  appropriateness of techniques for their contexts.
\item
  \textbf{Static documentation}: traditional publications and databases
  struggle to incorporate new techniques and update existing ones as the
  field evolves.
\item
  \textbf{Limited actionability}: academic surveys and framework
  documents, while comprehensive and rigorous, often lack the practical
  details needed for implementation (i.e.~favouring high-level and
  widely applicable guidance that fails to offer sufficient specificity
  to end users).
\end{enumerate}

By addressing these gaps through a combination of comprehensive
curation, systematic resource discovery, and accessible delivery
mechanisms, the TEA Techniques dataset seeks to provide a significant
source of extensible value to the broader community.

\section{TEA Techniques Dataset and App}\label{sec:dataset}

The TEA Techniques dataset represents a carefully designed repository
that seeks to balance comprehensive coverage with practical usability.
In this section, we describe the structure of the dataset and the
architecture of an interactive web app for navigating the dataset.

\subsection{Structure of the TEA Techniques
Dataset}\label{sec:structure}

The TEA Techniques dataset comprises 92 curated techniques organised
around a structured schema that seeks to balances comprehensiveness with
practical usability. Although the dataset is available in its entirety
as a single JSON file, each technique is also represented as a
self-contained JSON object.

\subsubsection{Technique Schema}\label{technique-schema}

Each technique in the database follows a consistent structure designed
to provide comprehensive yet accessible information, represented in the
below example:

\begin{Shaded}
\begin{Highlighting}[]
\FunctionTok{\{}
  \DataTypeTok{"name"}\FunctionTok{:} \StringTok{"SHapley Additive exPlanations"}\FunctionTok{,}
  \DataTypeTok{"acronym"}\FunctionTok{:} \StringTok{"SHAP"}\FunctionTok{,}
  \DataTypeTok{"description"}\FunctionTok{:} \StringTok{"SHAP explains model predictions by quantifying how much each input feature contributes to the outcome..."}\FunctionTok{,}
  \DataTypeTok{"assurance\_goals"}\FunctionTok{:} \OtherTok{[}\StringTok{"Explainability"}\OtherTok{,} \StringTok{"Fairness"}\OtherTok{,} \StringTok{"Reliability"}\OtherTok{]}\FunctionTok{,}
  \DataTypeTok{"tags"}\FunctionTok{:} \OtherTok{[}
    \StringTok{"applicable{-}models/agnostic"}\OtherTok{,}
    \StringTok{"assurance{-}goal{-}category/explainability/feature{-}analysis/}
\StringTok{     importance{-}and{-}attribution"}\OtherTok{,}
    \StringTok{"data{-}type/any"}\OtherTok{,}
    \StringTok{"evidence{-}type/quantitative{-}metric"}\OtherTok{,}
    \StringTok{"explanatory{-}scope/local"}\OtherTok{,}
    \StringTok{"explanatory{-}scope/global"}\OtherTok{,}
    \StringTok{"lifecycle{-}stage/model{-}development"}\OtherTok{,}
    \StringTok{"technique{-}type/algorithmic"}
  \OtherTok{]}\FunctionTok{,}
  \DataTypeTok{"example\_use\_cases"}\FunctionTok{:} \OtherTok{[}\FunctionTok{\{}
    \DataTypeTok{"description"}\FunctionTok{:} \StringTok{"Auditing a loan approval model..."}\FunctionTok{,}
    \DataTypeTok{"goal"}\FunctionTok{:} \StringTok{"Fairness"}
  \FunctionTok{\}}\OtherTok{]}\FunctionTok{,}
  \DataTypeTok{"limitations"}\FunctionTok{:} \OtherTok{[}
    \FunctionTok{\{}\DataTypeTok{"description"}\FunctionTok{:} \StringTok{"Assumes feature independence..."}\FunctionTok{\}}\OtherTok{,}
    \FunctionTok{\{}\DataTypeTok{"description"}\FunctionTok{:} \StringTok{"Computationally expensive..."}\FunctionTok{\}}
  \OtherTok{]}\FunctionTok{,}
  \DataTypeTok{"related\_techniques"}\FunctionTok{:} \OtherTok{[}
    \StringTok{"integrated{-}gradients"}\OtherTok{,}
    \StringTok{"local{-}interpretable{-}model{-}agnostic{-}explanations"}\OtherTok{,}
    \StringTok{"anchor"}
  \OtherTok{]}\FunctionTok{,}
  \DataTypeTok{"resources"}\FunctionTok{:} \OtherTok{[}
    \FunctionTok{\{}
      \DataTypeTok{"title"}\FunctionTok{:} \StringTok{"shap/shap"}\FunctionTok{,}
      \DataTypeTok{"url"}\FunctionTok{:} \StringTok{"https://github.com/shap/shap"}\FunctionTok{,}
      \DataTypeTok{"source\_type"}\FunctionTok{:} \StringTok{"software\_package"}
    \FunctionTok{\}}\OtherTok{,}
    \FunctionTok{\{}
      \DataTypeTok{"title"}\FunctionTok{:} \StringTok{"Introduction to SHAP — XAI Tutorials"}\FunctionTok{,}
      \DataTypeTok{"url"}\FunctionTok{:} \StringTok{"https://xai{-}tutorials.readthedocs.io/..."}\FunctionTok{,}
      \DataTypeTok{"source\_type"}\FunctionTok{:} \StringTok{"tutorial"}
    \FunctionTok{\}}
  \OtherTok{]}\FunctionTok{,}
  \DataTypeTok{"complexity\_rating"}\FunctionTok{:} \DecValTok{3}\FunctionTok{,}
  \DataTypeTok{"computational\_cost\_rating"}\FunctionTok{:} \DecValTok{4}
\FunctionTok{\}}
\end{Highlighting}
\end{Shaded}

The \textbf{description} provides a clear, accessible explanation of the
technique's core concept and operation, avoiding excessive jargon while
seeking to maintain technical accuracy. This enables stakeholders with
varying technical backgrounds to understand the technique's purpose and
approach.

The \textbf{assurance goals} array identifies which specific goals the
technique supports from the seven core assurance goals detailed in
Section \ref{sec:goals}. Techniques commonly address multiple goals
simultaneously. For instance, SHAP provides explainability while also
supporting fairness assessments by revealing how protected
characteristics influence predictions.

The \textbf{tags} array provides structured metadata enabling precise
filtering and discovery. Tags follow a hierarchical prefix system (see
Section \ref{sec:tags}) covering dimensions such as applicable model
types, data requirements, lifecycle stages, and expertise needed.

\textbf{Example use cases} illustrate concrete applications across
different domains, helping practitioners envision how the technique
might apply to their specific contexts. Each use case is tied to a
specific assurance goal, demonstrating the evidence the technique can
provide for its respective goals.

\textbf{Limitations} are explicitly documented as separate points,
ensuring practitioners understand the technique's constraints,
assumptions, and potential failure modes. This transparency is crucial
for building robust assurance arguments that acknowledge uncertainties.

\textbf{Related techniques} connect each method to alternatives with
different strengths, trade-offs, and applicability contexts. Alongside,
the limitations section, this structured comparison set also helps
practitioners discover more suitable options rather than defaulting to
familiar approaches (i.e.~challenging habitual assumptions).

\subsubsection{Encouraging Deliberate Technique
Selection}\label{encouraging-deliberate-technique-selection}

The explicit documentation of limitations and related techniques serves
a critical function in promoting evidence-based technique selection over
habitual use of familiar methods. Research in cognitive psychology
demonstrates that practitioners often exhibit confirmation bias and
availability heuristic effects, leading them to repeatedly apply
techniques they know well regardless of contextual appropriateness.

By prominently displaying limitations, the dataset forces practitioners
to confront potential weaknesses before implementation. For instance,
SHAP's computational expense and feature independence assumptions may
make LIME or Permutation Feature Importance more suitable for certain
contexts. The related techniques section then provides immediate
alternatives with clear relationship descriptions, enabling
practitioners to compare trade-offs systematically rather than
proceeding with suboptimal but familiar choices.

This design philosophy recognises that technique selection should be
driven by contextual requirements rather than practitioner familiarity,
ultimately improving the quality and appropriateness of AI assurance
implementations across diverse applications.

\textbf{Resources} link to high-quality external materials including
software implementations, documentation, tutorials, and foundational
papers. Each resource includes a description and type classification,
enabling targeted exploration.

The methodology for discovering and curating these resources is detailed
in Section \ref{sec:discovery}.

\textbf{Ratings} for complexity and computational cost provide quick
indicators of implementation difficulty and resource requirements,
supporting feasibility assessment during technique selection.

These subjective ratings require ongoing refinement through community
feedback mechanisms to ensure accuracy and consistency across diverse
implementation contexts, as discussed in Section \ref{sec:future}.

\subsubsection{Core Assurance Goals}\label{sec:goals}

The primary means for organising the techniques is around 7 core
assurance goals, each addressing critical aspects of trustworthy and
ethical AI:

\begin{itemize}
\tightlist
\item
  \textbf{Explainability}: techniques for understanding and interpreting
  AI system's and their behaviour, ranging from feature importance
  analysis to counterfactual generation. These techniques address a
  fundamental need to understand and explain how AI systems reach their
  decisions, and are essential for processes such as debugging,
  improvement, and stakeholder trust and engagement.
\item
  \textbf{Fairness}: techniques for identifying, measuring, and
  mitigating various forms of bias and discrimination in AI systems and
  the data upon which they are trained. The techniques span lifecycle
  stages such as pre-processing, in-processing, and post-processing
  methods, as well as different types of fairness (e.g.~group or
  individual).
\item
  \textbf{Privacy}: techniques for protecting individual privacy while
  still enabling AI functionality, including differential privacy
  implementations, federated learning approaches, and privacy-preserving
  computation techniques.
\item
  \textbf{Reliability}: techniques for ensuring consistent, dependable
  AI performance across diverse conditions or deployments, including
  robustness testing, uncertainty quantification, and failure mode
  analysis.
\item
  \textbf{Safety}: techniques for preventing AI systems from causing
  harm (including physical, mental, socioeconomic, and environmental),
  such as adversarial testing, safety verification, and containment
  strategies for high-risk applications.
\item
  \textbf{Transparency}: techniques for making AI systems and their
  developmental and organisational processes more open and
  understandable, including documentation standards, audit trails, and
  communication frameworks.
\item
  \textbf{Security}: techniques for protecting AI systems from malicious
  attacks, unauthorised access, and exploitation vulnerabilities,
  including adversarial defense mechanisms, secure computing approaches
  like homomorphic encryption, and red teaming methodologies for
  proactive vulnerability assessment.
\end{itemize}

It is important to note, however, that while these core goals help
direct the focus of assurance cases, many techniques may serve multiple
goals---especially in the generation of evidential artefacts. For
instance, model cards contribute to both transparency and fairness by
documenting system characteristics and performance across different
populations; integrated gradients can be used to both interpret and
explain predictions made by AI and also to improve safety and
reliability; and concept activation vectors can help identify sources of
bias due to the use of protected characteristics as well as more obvious
explainability and transparency purposed. The dataset seeks to capture
these cross-cutting relationships, enabling practitioners to identify
techniques that address multiple assurance objectives simultaneously.

\subsubsection{Tagging System}\label{sec:tags}

The database employs a flexible tagging system to help further organise
techniques into hierarchical categories. This enables more precise
filtering and discovery of techniques.

Tags employ a hierarchical prefix structure where categories are
separated by forward slashes, enabling both broad filtering (e.g., all
``explainability'' techniques) and precise targeting (e.g.,
``explainability/feature-analysis/importance-and-attribution'').\footnote{Complete
  tag definitions and examples are available at
  https://alan-turing-institute.github.io/tea-techniques/about/tag-definitions/}

\begin{longtable}[]{@{}
  >{\raggedright\arraybackslash}p{(\linewidth - 4\tabcolsep) * \real{0.3333}}
  >{\raggedright\arraybackslash}p{(\linewidth - 4\tabcolsep) * \real{0.3333}}
  >{\raggedright\arraybackslash}p{(\linewidth - 4\tabcolsep) * \real{0.3333}}@{}}
\caption{Tag Categories and Their Descriptions}\tabularnewline
\toprule\noalign{}
\begin{minipage}[b]{\linewidth}\raggedright
Tag Category
\end{minipage} & \begin{minipage}[b]{\linewidth}\raggedright
Description
\end{minipage} & \begin{minipage}[b]{\linewidth}\raggedright
Example Tags
\end{minipage} \\
\midrule\noalign{}
\endfirsthead
\toprule\noalign{}
\begin{minipage}[b]{\linewidth}\raggedright
Tag Category
\end{minipage} & \begin{minipage}[b]{\linewidth}\raggedright
Description
\end{minipage} & \begin{minipage}[b]{\linewidth}\raggedright
Example Tags
\end{minipage} \\
\midrule\noalign{}
\endhead
\bottomrule\noalign{}
\endlastfoot
\textbf{applicable-models} & Specifies which model architectures the
technique supports & \texttt{model-agnostic},
\texttt{neural-networks} \\
\textbf{assurance-goal-category} & Fine-grained categorisation within
each assurance goal & \texttt{explainability/feature-analysis},
\texttt{fairness/measurement} \\
\textbf{data-requirements} & Special data needs such as labelled data or
temporal sequences & \texttt{labelled-data},
\texttt{temporal-sequences} \\
\textbf{data-type} & Applicable data modalities (tabular, text, image,
etc.) & \texttt{tabular}, \texttt{text}, \texttt{image} \\
\textbf{evidence-type} & Type of output produced (metrics,
visualisations, reports) & \texttt{quantitative-metrics},
\texttt{visualizations} \\
\textbf{expertise-needed} & Required knowledge domains &
\texttt{statistics}, \texttt{causal-inference} \\
\textbf{explanatory-scope} & Local (instance) vs global (model)
explanations & \texttt{local}, \texttt{global} \\
\textbf{lifecycle-stage} & AI development phases &
\texttt{model-development}, \texttt{deployment} \\
\textbf{technique-type} & Nature of the technique &
\texttt{algorithmic}, \texttt{analytical} \\
\end{longtable}

Complete tag definitions and examples are available in the
\href{https://alan-turing-institute.github.io/tea-techniques/about/tag-definitions/}{web
application}.

Additional category-specific tags capture unique dimensions. For
instance, fairness techniques include tags for individual vs.~group
fairness approaches, while explainability techniques distinguish between
local or global explanation strategies.\footnote{Local explanations
  focus on understanding individual predictions (e.g., why this specific
  loan was rejected), while global explanations reveal overall model
  behaviour patterns (e.g., how the lending model generally weighs
  different factors).}

This tagging system has been designed to enable precise search and
filtering functionality, helping to reveal relationships between
techniques. However, unlike a more rigid schema, it also is intended to
be updated and enhanced over time (e.g.~adding new categories and tags).
Possible future options could include a ranking system of popular tags,
or merging and splitting of tags that are under- or over-used.

\subsubsection{Evaluation Criteria}\label{evaluation-criteria}

So far we have set aside the question of ``what is a technique?''

However, this is an important question to address, as the answer can
also be used as a means to demarcate relevant from non-relevant
techniques, or to exclude ill-defined or poor quality methods.

In the context of the TEA Techniques dataset and web app, a
``technique'' is defined as a \textbf{concrete method that produces
tangible evidence for AI assurance claims}. This definition deliberately
excludes abstract principles, high-level guidelines, or purely
theoretical constructs that lack practical implementation methods.

\paragraph{Core Requirements for
Techniques}\label{core-requirements-for-techniques}

Each technique in our database must satisfy four foundational criteria:

\begin{enumerate}
\def\labelenumi{\arabic{enumi}.}
\item
  \textbf{Evidence Generation}: The technique must produce specific
  outputs (metrics, visualisations, reports, proofs) that can serve as
  evidence within argument-based assurance frameworks. For example, SHAP
  produces feature attribution values that evidence explainability
  claims.
\item
  \textbf{Practical Applicability}: Practitioners with appropriate
  expertise must be able to implement the technique using available
  tools and resources. We include emerging techniques if they have
  reference implementations or detailed methodological descriptions.
\item
  \textbf{Direct Relevance}: The technique must explicitly address one
  or more of our six assurance goals (explainability, fairness, privacy,
  reliability, safety, transparency). Tangentially related methods are
  excluded unless they produce evidence directly supporting these goals.
\item
  \textbf{Quality Threshold}: Techniques must demonstrate sufficient
  maturity through peer review, empirical validation, or substantial
  community adoption. Experimental methods may be included with
  appropriate caveats if they show significant promise.
\end{enumerate}

\paragraph{Evaluation Framework}\label{evaluation-framework}

We evaluate candidate techniques across eight dimensions to ensure
consistent quality and relevance:

\begin{itemize}
\tightlist
\item
  \textbf{Goal alignment}: How directly the technique addresses
  assurance objectives
\item
  \textbf{Evidence type}: The nature and strength of outputs produced
  (quantitative metrics, visualisations, formal proofs)
\item
  \textbf{Applicability scope}: Model types, data formats, and lifecycle
  stages where the technique applies
\item
  \textbf{Maturity level}: Publication record, empirical validation, and
  community adoption
\item
  \textbf{Resource requirements}: Computational costs, data needs, and
  expertise prerequisites
\item
  \textbf{Actionability}: How readily outputs translate into concrete
  decisions or interventions
\item
  \textbf{Limitations}: Documented assumptions, failure modes, and scope
  boundaries
\item
  \textbf{Specificity}: The technique's precision in addressing
  particular assurance challenges
\end{itemize}

This multi-dimensional evaluation ensures the database includes
techniques that are both theoretically sound and practically valuable,
supporting evidence-based approaches to responsible AI development
across diverse contexts and applications.

\subsection{TEA Techniques Web App}\label{sec:webapp}

While there may be value in the raw JSON file for some practitioners,
this format is not suitable for others. And, as one of our key goals was
to widen participation in the AI assurance ecosystem, we decided to
develop an interactive platform for navigating this dataset.

The TEA Techniques web application provides an intuitive interface for
exploring the complete dataset of AI assurance techniques. Built as a
static Next.js application and hosted on GitHub Pages, the platform
offers multiple pathways for discovering relevant techniques through
both browsing and searching capabilities. Figure \ref{fig:fairness-page}
shows the main category page interface, while Figure
\ref{fig:technique-page} displays a detailed technique view.

\begin{figure}
\centering
\pandocbounded{\includegraphics[keepaspectratio,alt={Fairness Category Page displaying techniques organized by assurance goal with filtering sidebar and search functionality}]{figures/fairness-category-page.png}}
\caption{Fairness Category Page displaying techniques organized by
assurance goal with filtering sidebar and search
functionality}\label{fig:fairness-page}
\end{figure}

Users can navigate the dataset through several entry points: browsing by
assurance goal categories, filtering by specific tags, using the
full-text search functionality (accessible via Cmd+K), or exploring
individual technique pages for detailed information. Each technique page
provides comprehensive details including descriptions, use cases,
limitations, related techniques, and curated external resources.

\begin{figure}
\centering
\pandocbounded{\includegraphics[keepaspectratio,alt={Integrated Gradients Technique Page showing detailed information including description, example use cases, limitations, resources, and related techniques}]{figures/integrated-gradients-technique-page.png}}
\caption{Integrated Gradients Technique Page showing detailed
information including description, example use cases, limitations,
resources, and related techniques}\label{fig:technique-page}
\end{figure}

The web application's responsive design ensures accessibility across
devices, while the static architecture guarantees fast load times and
reliability. The platform serves both as a reference tool for
practitioners seeking specific techniques and as an educational resource
for those learning about AI assurance methodologies.

The interactive web app's architecture reflects several key design
principles:

\begin{itemize}
\tightlist
\item
  \textbf{Practitioner-centricity}: we aim to prioritise the needs of
  practitioners who must select and implement techniques in real-world
  contexts.
\item
  \textbf{Evidence-orientation}: techniques are presented as means for
  supporting evidence generation---specifically (but not solely) in the
  production of assurance cases.
\item
  \textbf{Comprehensive and extensible metadata}: structured metadata
  and flexible tagging enables multi-dimensional filtering of the
  dataset (e.g.~applicable model type, data characteristics, expertise
  needed, and lifecycle stage).
\item
  \textbf{Usability and Transparency}: clear documentation of technique
  limitations, requirements, and complexity indicators help
  practitioners make informed decisions about techniques prior to
  selection or adoption.
\end{itemize}

The web application employs a static site architecture hosted on GitHub
Pages. While this limits functionality that would be dependant on the
existence of an API and database, it helps ensure sustainable access
without ongoing infrastructure costs that could compromise the
platform's community-centred mission.

Static pages are generated at build time (using GitHub actions), using
the primary dataset as the source, and enabling fast loading while
maintaining the flexibility for complex filtering and search
functionality.

\section{Technique and Resource Discovery}\label{sec:discovery}

\subsection{Initial Set of Techniques}\label{initial-set-of-techniques}

\colorbox{yellow}{TODO: Explain non-systematic method for initial set (internal testing and feedback)}

\subsection{Resource Discovery
Pipeline}\label{resource-discovery-pipeline}

The longer term value of the TEA Techniques dataset and app will depend
upon a) ongoing maintenance by the core team, and b) how well it serves
the need of an engaged community. For instance, promoting submission of
new techniques by members of the AI assurance ecosystem; encouraging
active discussion and feedback around content quality; and maintaining
core architecture and reliability.

The success of digital commons projects, such as Wikipedia or
OpenStreetMap, demonstrates that community participation is essential
for maintaining currency, expanding coverage and adoption, and ensuring
quality and value. The TEA Techniques dataset and app will follow this
model (see Section \ref{sec:future}).

However, all community-centred projects need an initial starting point
to help found an active community of practice. To that end, we carried
out a systematic search for an initial set of resources to support the
preliminary techniques.

In this section we explain the methodology and tooling used to support
the systematic identification, evaluation, and categorisation of
supporting resources.

\subsubsection{Pipeline Overview}\label{pipeline-overview}

The resource discovery pipeline we used followed a seven-stage
sequential processing model, summarised in Table 2.

\begin{longtable}[]{@{}
  >{\raggedright\arraybackslash}p{(\linewidth - 4\tabcolsep) * \real{0.2188}}
  >{\raggedright\arraybackslash}p{(\linewidth - 4\tabcolsep) * \real{0.2812}}
  >{\raggedright\arraybackslash}p{(\linewidth - 4\tabcolsep) * \real{0.5000}}@{}}
\caption{Resource Discovery Pipeline Stages}\tabularnewline
\toprule\noalign{}
\begin{minipage}[b]{\linewidth}\raggedright
Stage
\end{minipage} & \begin{minipage}[b]{\linewidth}\raggedright
Purpose
\end{minipage} & \begin{minipage}[b]{\linewidth}\raggedright
Key Activities
\end{minipage} \\
\midrule\noalign{}
\endfirsthead
\toprule\noalign{}
\begin{minipage}[b]{\linewidth}\raggedright
Stage
\end{minipage} & \begin{minipage}[b]{\linewidth}\raggedright
Purpose
\end{minipage} & \begin{minipage}[b]{\linewidth}\raggedright
Key Activities
\end{minipage} \\
\midrule\noalign{}
\endhead
\bottomrule\noalign{}
\endlastfoot
PREPARE & Query Generation & Prepares technique data for
platform-specific searches \\
SEARCH & Multi-Source Discovery & Searches 5 platforms for up to 75
candidates per technique \\
RANK & Quality Selection & Applies 6-factor scoring to select top 5-10
resources \\
EVALUATE & LLM Assessment & Evaluates resources for relevance and
refines classifications \\
VALIDATE & Accessibility Check & Verifies URLs are accessible and
extracts metadata \\
DEDUPLICATE & Remove Redundancy & Eliminates duplicates through URL and
content comparison \\
SAVE & Persistent Storage & Stores final curated resources with audit
trail \\
\end{longtable}

\subsubsection{Prepare and Search}\label{prepare-and-search}

Because each technique was run individually due to limits on API access
for the search repositories, the initial PREPARE stage comprised a
simple data processing step, extracting relevant data (e.g.~title,
acronym, description) from the original \texttt{techniques.json} file.

The SEARCH stage is particularly important as it queries five distinct
repositories to ensure comprehensive coverage:

\begin{itemize}
\tightlist
\item
  \textbf{Academic sources} (arXiv, Semantic Scholar, CORE): Up to 45
  papers focusing on theoretical foundations and peer-reviewed research
\item
  \textbf{GitHub}: Up to 10 well-maintained repositories (minimum 10
  stars) for practical implementations
\item
  \textbf{Google Custom Search}: Up to 20 web resources including
  tutorials and documentation from relevant sites (e.g.~HuggingFace,
  Readthedocs.io, Kaggle, Towards Data Science)
\end{itemize}

This multi-source approach typically yields 30-75 initial candidates per
technique, which are then balanced to ensure diversity across resource
types and sources.

\subsubsection{LLM Evaluation and
Ranking}\label{llm-evaluation-and-ranking}

The EVALUATE and RANK stages leverage large language models
(specifically Anthropic's Claude Sonnet 3.7) to ensure only pertinent
results advance through the pipeline. This approach was designed to a)
automate the process of resource discovery at scale, and b) minimise
false positives that traditional keyword matching produces.

For instance, searches for ``SHAP'' (i.e.~Shapley Additive Explanations)
would often return results about the popular ``Shapez'' video game.
Using LLM evaluation effectively filtered out such irrelevant matches,
while preserving genuinely related resources about SHAP (SHapley
Additive exPlanations). All original search results were saved to enable
manual verification of the filtering process.

For the RANK stage, the pipeline applied a six-factor quality scoring
system that considers source credibility (30\%), content recency (20\%),
popularity/impact (20\%), and other quality metrics. This also helped
constrain the LLM decision-making to minimise inaccuracies.

\subsubsection{Validation and SAVE}\label{validation-and-save}

The VALIDATE and SAVE stages ensured all selected resources were
accessible, following redirect chains and verifying content
availability. This prevents the inclusion of dead links or resources
that have moved, maintaining the dataset's reliability for end users.

\subsubsection{Human Review Process}\label{human-review-process}

Because of the use of LLM-powered evaluation and ranking, after the
automated pipeline processed the initial set of 92 techniques,
\textbf{all of the resources} were manually reviewed by a human to
ensure that they were relevant to the respective technique. This final
assurance step is an important first quality gate, but as we will
discuss later, there are additional plans for implementing community
feedback mechanisms to further enhance the quality of resources.

\subsection{Results and Impact}\label{results-and-impact}

The pipeline's systematic approach transformed an initial pool of
approximately 2,760-6,900 candidate resources (30-75 per technique
across 92 techniques) into a carefully curated dataset of
\textasciitilde360 high-quality resources. This represents an average of
approximately 4 resources per technique, with each resource verified for
accessibility, relevance, and quality through both automated scoring and
human review.

\section{Web Application Technical Details}\label{sec:technical}

TEA Techniques employs a static site generation (SSG) architecture using
Next.js 14.

\subsection{Static Data Generation}\label{static-data-generation}

The build process transforms the master \texttt{techniques.json} file
into optimised data structures for different access patterns:

\begin{itemize}
\tightlist
\item
  \textbf{Individual technique files}: Each technique gets a dedicated
  JSON file enabling direct access without parsing the entire dataset.
\item
  \textbf{Category indices}: Pre-computed lists of techniques for each
  assurance goal, tag, and tag combination reduce client-side filtering.
\item
  \textbf{Metadata aggregation}: Statistics, tag counts, and
  relationship mappings are pre-calculated for immediate display.
\end{itemize}

\subsection{TEA Techniques ``API''}\label{tea-techniques-api}

Despite being a static site, TEA Techniques provides API-like access to
our dataset.

All JSON files are publicly accessible through predictable URL patterns
(i.e. slugs based on the technique's name), enabling programmatic access
to:

\begin{itemize}
\tightlist
\item
  Complete dataset: \texttt{/data/techniques.json}
\item
  Individual techniques: \texttt{/data/techniques/{[}slug{]}.json}
\item
  Techniques by goal:
  \texttt{/data/categories/{[}goal{]}/techniques.json}
\item
  Techniques by tag: \texttt{/data/tags/{[}tag{]}/techniques.json}
\end{itemize}

The URL structure supports future versioning without breaking existing
integrations, ensuring stability for tools and workflows that depend on
the API endpoints.

Development support for the community also includes a Docker compose
configuration available on GitHub to ensure consistent development
experiences for community contributors. Comprehensive documentation and
explanatory pages within the web application explain key concepts and
operational details. The platform maintains user privacy by avoiding
cookies and personal identifiers, ensuring technique discovery remains
anonymous and accessible.

\section{Use Cases and Applications}\label{sec:usecases}

The true value of TEA Techniques emerges through its application in
real-world scenarios. In this section, we propose a couple of possible
use cases and workflows, demonstrating how the platform could be
leveraged to support AI assurance activities in different settings.

\subsection{Practitioner Workflows}\label{practitioner-workflows}

Development teams building AI systems face constant decisions about
which assurance techniques to implement. TEA Techniques supports their
workflow through several mechanisms:

\textbf{Technique selection during design}: Teams can filter techniques
by lifecycle stage to identify relevant methods for their current phase.
For instance, a team designing a loan approval system might filter for
fairness techniques applicable during model development, discovering
options like fairness-aware preprocessing, in-processing constraints,
and post-processing calibration.

\textbf{Implementation guidance}: Each technique's resources section
provides immediate access to code libraries and implementation guides. A
team implementing SHAP explanations can quickly find the official
library, integration examples for their framework (TensorFlow, PyTorch,
scikit-learn), and tutorials addressing common pitfalls.

\textbf{Cross-functional collaboration}: The accessible technique
descriptions enable productive conversations between technical and
non-technical team members. Product managers can understand the evidence
each technique provides, whilst engineers focus on implementation
details.

\subsection{Regulatory Compliance and Risk
Officers}\label{regulatory-compliance-and-risk-officers}

Organisations must demonstrate their AI systems meet regulatory
requirements and organisational policies:

\textbf{Regulatory alignment}: Techniques can be mapped to specific
regulatory requirements. For instance, GDPR's requirement for meaningful
explanations aligns with local explanation techniques, whilst fairness
regulations map to statistical parity tests and disparate impact
assessments.

\textbf{Risk assessment}: The limitations and assumptions documented for
each technique help risk officers understand potential gaps in their
assurance approach. Knowing that LIME assumes local linearity, for
example, prompts consideration of complementary techniques for highly
non-linear models.

\textbf{Policy development}: The comprehensive technique inventory
informs organisational policy development by revealing the art of the
possible---what types of assurance evidence can reasonably be required
given available methods.

\subsection{AI Research Projects}\label{ai-research-projects}

TEA Techniques serves as a comprehensive resource for both education and
research. For instance, instructors can structure courses around the six
assurance goals, using the database to ensure comprehensive coverage of
available techniques. Students can implement and compare different
techniques using the provided resources, moving beyond theoretical
discussion to hands-on experience. And, researchers can identify gaps in
the current technique landscape as starting points for research projects
or dissertations.

As the community of practice develops and usage patterns emerge, the TEA
Techniques platform could evolve into a valuable research data
repository. Aggregated usage analytics, technique adoption patterns, and
community feedback could provide insights into how AI assurance methods
are being applied across different domains and contexts, offering
empirical foundations for new research programmes investigating the
effectiveness and evolution of responsible AI practices.

\colorbox{yellow}{TODO: maybe discuss self-paced learning or certification routes}

\section{Community Engagement, Extensibility, and Future
Work}\label{sec:future}

\subsection{Community Engagement}\label{community-engagement}

The launch of TEA Techniques represents not an endpoint but the
beginning of a community-driven evolution in AI assurance practice.

Initially, community contributions will be managed through GitHub Issues
for technique suggestions and discussions, with Pull Requests handling
formal submissions and revisions. This approach leverages existing
version control capabilities for transparent review processes while
maintaining quality standards. However, this mechanism will need to be
monitored for accessibility barriers (i.e.~prior knowledge of Git and
GitHub), and alternative submission pathways may be introduced if
GitHub's technical requirements exclude valuable community participation
from non-technical stakeholders.

\colorbox{yellow}{TODO: explain QA pipeline - automated validation and human review}

\colorbox{yellow}{TODO: explain attribution and contributor recognition system}

\subsection{Extensibility}\label{extensibility}

A key extensibility priority involves integration with the
\href{https://assuranceplatform.azurewebsites.net/}{TEA Platform},
enabling seamless technique discovery during assurance case development.
This integration will allow practitioners to identify relevant
techniques contextually while building arguments, and discover how
others have applied specific techniques in their own assurance cases.
Such integration transforms TEA Techniques from a standalone resource
into an active component of the assurance development workflow.

\subsubsection{Standards Alignment}\label{standards-alignment}

\colorbox{yellow}{TODO: speak with AISH team about standardisation engagement}

\begin{itemize}
\tightlist
\item
  \textbf{ISO/IEC collaboration}: Alignment with emerging AI standards
  (ISO/IEC 23053, 23894) ensuring technique categorisations support
  compliance demonstration.
\item
  \textbf{Regulatory mapping}: Explicit mapping between techniques and
  regulatory requirements across jurisdictions, supporting global
  compliance efforts.
\item
  \textbf{Industry frameworks}: Integration with industry-specific
  frameworks (IEEE, partnership on AI) promoting consistent terminology
  and approach.
\end{itemize}

\subsection{Future Plans}\label{future-plans}

\subsubsection{Enhanced Search and
Discovery}\label{enhanced-search-and-discovery}

Current search capabilities, whilst functional, have room for
significant enhancement:

\begin{itemize}
\tightlist
\item
  \textbf{Semantic search}: Integration of embedding-based search will
  enable conceptual queries---finding techniques similar to a described
  need rather than relying on keyword matching.
\item
  \textbf{Faceted exploration}: Advanced filtering interfaces will
  support complex queries combining multiple criteria with AND/OR logic,
  enabling precise technique discovery for specific contexts.
\end{itemize}

\subsubsection{Interactive Features}\label{interactive-features}

Possible features for enabling greater interactivity and usability could
include:

\begin{itemize}
\tightlist
\item
  \textbf{Technique comparison tools}: Side-by-side comparison
  interfaces will help users evaluate trade-offs between similar
  techniques, with visualisations highlighting key differences.
\item
  \textbf{Decision support wizards}: Interactive questionnaires will
  guide users through technique selection based on their specific
  context, constraints, and objectives.
\item
  \textbf{Integration playgrounds}: Sandbox environments could allow
  users to experiment with technique implementations using sample
  datasets, lowering barriers to adoption.
\end{itemize}

\subsubsection{Content Expansion}\label{content-expansion}

As AI technology evolves, new techniques will emerge as will new
categories. There are currently many gaps in our dataset, as well as
significant imbalance between categories.

Our analysis reveals substantial imbalances across assurance goals, as
illustrated in Figure 3. Here you can see that Fairness dominates with
53 techniques, followed closely by Reliability and Transparency (each
with 52 techniques, many of these overlapping). In contrast, Security
represents only 3 techniques, creating a highly imbalanced preliminary
set of techniques, as illustrated in Figure \ref{fig:distribution}.

\begin{figure}
\centering
\pandocbounded{\includegraphics[keepaspectratio,alt={Distribution of TEA Techniques by Assurance Goal: Bar chart showing the number of techniques per assurance goal, with Fairness (53), Reliability (52), and Transparency (52) dominating the dataset, while Privacy (7) and Security (3) remain severely underrepresented.}]{figures/assurance_goals_distribution.png}}
\caption{Distribution of TEA Techniques by Assurance Goal: Bar chart
showing the number of techniques per assurance goal, with Fairness (53),
Reliability (52), and Transparency (52) dominating the dataset, while
Privacy (7) and Security (3) remain severely
underrepresented.}\label{fig:distribution}
\end{figure}

Moreover, novel and emerging techniques (e.g.~those specific to large
language models, image generators, and other generative AI systems)
require dedicated coverage addressing unique challenges like
hallucination detection and prompt injection prevention. This is a known
gap in the current dataset that we intend to prioritise going forward.

Domain-specific technique collections could also be added to help
address sector needs (e.g.~healthcare AI assurance techniques addressing
clinical validation, patient safety, and regulatory compliance specific
to medical AI applications).

And, finally, expanding access to a global audience requires
multilingual capabilities (e.g.~interface localisation, technique
translation, and regional resource curation).{]}

\subsubsection{Open Challenges}\label{open-challenges}

Several methodological challenges require ongoing attention as the
platform evolves. Managing technique obsolescence presents a complex
balance between maintaining historical accuracy and providing current
guidance. Clear deprecation criteria must be developed that can identify
when techniques become outdated due to superseding methods or changing
technical standards, while still preserving the original data records.

The tension between quality and inclusivity also poses another
significant challenge. Rigorous evaluation standards will, hopefully,
ensure practitioners receive reliable guidance, yet overly strict
criteria may exclude promising experimental techniques that could
benefit from community feedback and iterative improvement. A potential
solution involves implementing community-based maturity scoring systems
that provide transparency about technique readiness while enabling
incremental quality enhancement through collective assessment.

Establishing reliable evaluation metrics also remains problematic given
the subjective nature of complexity and computational cost assessments.
We recognise that standardised (and accepted) measurement approaches are
needed that can meaningfully compare techniques across different
implementation contexts, computing environments, and skill requirements.
Such metrics must be both objective enough to support systematic
comparison and flexible enough to account for the diverse contexts in
which techniques are applied.

\section{Conclusion}\label{sec:conclusion}

The TEA Techniques platform represents a significant step forward in
widening access and participation in the AI assurance ecosystem. By
combining technique curation with systematic resource discovery and
accessible delivery mechanisms, we have aimed to create a
community-centred resource that bridges the gap between high-level
principles and practical implementation.

The impact of the dataset and web app impact extends across multiple
dimensions. For practitioners, it provides immediate access to
\emph{actionable techniques} with clear guidance and documentation. For
researchers, it offers a structured landscape of existing methods, and
the possibility of a community-driven repository that could offer
additional research opportunities (e.g.~which resources are highly used,
where are there current gaps to be addressed). For educators and
policymakers, it supplies concrete examples of what AI assurance
entails, moving beyond abstract and high-level frameworks to specific,
implementable methods.

Our resource discovery pipeline offers an early demonstration of how
LLM-powered automation and human judgment can be combined effectively to
manage information at scale. The programmatic filtering combined with
the stochastic LLM-based evaluation process ensure that practitioners
receive not just any resources, but those most likely to support
successful technique implementation. This methodology could be adapted
to other domains facing similar challenges of information overload and
quality uncertainty.

Perhaps most importantly, the TEA Techniques dataset embodies the
philosophy of openness and accessibility---aiming to lower barriers to
trustworthy and ethical AI assurance. The platform's current set of 92
techniques provides a solid foundation, but its true potential lies in
supporting a flourishing community of practice. As practitioners apply
techniques and share experiences, as researchers develop new methods,
and as regulations evolve to require specific assurances, the TEA
Techniques dataset will grow to meet these needs.

\section*{References}\label{bibliography}
\addcontentsline{toc}{section}{References}

\protect\phantomsection\label{refs}
\begin{CSLReferences}{1}{0}
\bibitem[\citeproctext]{ref-barocas2019fairness}
Barocas, Solon, Moritz Hardt, and Arvind Narayanan. 2019. \emph{Fairness
and Machine Learning}. fairmlbook.org.

\bibitem[\citeproctext]{ref-brundage2020toward}
Brundage, Miles, Shahar Avin, Jasmine Wang, Haydn Belfield, Gretchen
Krueger, Gillian Hadfield, Heidy Khlaaf, et al. 2020. {`Toward
Trustworthy AI Development: Mechanisms for Supporting Verifiable
Claims'}. \emph{arXiv:2004.07213 {[}Cs{]}}.
\url{http://arxiv.org/abs/2004.07213}.

\bibitem[\citeproctext]{ref-burr2022ethical}
Burr, Christopher, and David Leslie. 2022. {`Ethical Assurance: A
Practical Approach to the Responsible Design, Development, and
Deployment of Data-Driven Technologies'}. \emph{AI and Ethics}.
\url{https://doi.org/10.1007/s43681-022-00178-0}.

\bibitem[\citeproctext]{ref-linardatos2020explainable}
Linardatos, Pantelis, Vasilis Papastefanopoulos, and Sotiris Kotsiantis.
2020. {`Explainable AI: A Review of Machine Learning Interpretability
Methods'}. \emph{Entropy} 23 (1): 18.
\url{https://doi.org/10.3390/e23010018}.

\bibitem[\citeproctext]{ref-lundberg2017unified}
Lundberg, Scott M., and Su-In Lee. 2017. {`A Unified Approach to
Interpreting Model Predictions'}. In \emph{Advances in Neural
Information Processing Systems 30}, 4765--74.
\url{https://papers.nips.cc/paper/7062-a-unified-approach-to-interpreting-model-predictions}.

\bibitem[\citeproctext]{ref-mehrabi2019survey}
Mehrabi, Ninareh, Fred Morstatter, Nripsuta Saxena, Kristina Lerman, and
Aram Galstyan. 2019. {`A Survey on Bias and Fairness in Machine
Learning'}. \emph{arXiv:1908.09635 {[}Cs{]}}.
\url{http://arxiv.org/abs/1908.09635}.

\bibitem[\citeproctext]{ref-raji2020closing}
Raji, Inioluwa Deborah, Andrew Smart, Rebecca N White, Margaret
Mitchell, Timnit Gebru, Ben Hutchinson, Jamila Smith-Loud, Daniel
Theron, and Parker Barnes. 2020. {`Closing the AI Accountability Gap:
Defining an End-to-End Framework for Internal Algorithmic Auditing'},
12.

\bibitem[\citeproctext]{ref-ribeiro2016should}
Ribeiro, Marco Tulio, Sameer Singh, and Carlos Guestrin. 2016. {`"Why
Should i Trust You?": Explaining the Predictions of Any Classifier'}. In
\emph{Proceedings of the 22nd ACM SIGKDD International Conference on
Knowledge Discovery and Data Mining}, 1135--44.
\url{https://doi.org/10.1145/2939672.2939778}.

\end{CSLReferences}

\end{document}
